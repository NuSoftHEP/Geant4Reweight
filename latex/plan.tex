\setlength{\headheight}{15pt}
\documentclass[12pt]{article}
\usepackage{fancyhdr}
\lhead{}
\chead{}
\rhead{}
\renewcommand{\headrulewidth}{0pt}
\pagestyle{fancy}
\usepackage{graphicx}
\usepackage[top=2cm,bottom=3cm]{geometry}
\usepackage[svgnames]{xcolor}
\usepackage[colorlinks=true,linkcolor=DarkBlue,citecolor=DarkBlue]{hyperref}
\usepackage{xspace}
\usepackage{rotating}
\usepackage{units}
%\usepackage{subfig}
%\usepackage{amssymb, amsmath}
\usepackage{amsmath}
\usepackage{authblk}
\usepackage{lineno}
\usepackage{listings} 
\usepackage[normalem]{ulem}
\usepackage{adjustbox}
%\usepackage{placeins}
\usepackage[section]{placeins}
\usepackage{qtree}
\usepackage{SIunits}
\usepackage{hepunits}
\usepackage{hepparticles}
\usepackage{cancel}
\usepackage{hepnames}
\usepackage{epstopdf}
\usepackage{mathtools}
\usepackage{caption}
\usepackage[aboveskip=-10pt]{subcaption}
\usepackage[capitalise]{cleveref}
\usepackage{braket}
\usepackage{slashed}
\usepackage{subfiles}
\usepackage{graphicx}
\usepackage{textcomp}
\newcommand{\textapprox}{\raisebox{0.5ex}{\texttildelow}}

\newcommand{\todo}[1]{{\color{red} TODO: #1}}
\newcommand\red[1]{{\color{red}#1}}
\newcommand{\ccpi}{CC1$\pi^0$\xspace}
\newcommand{\ccpis}{CC$\pi^0$\xspace}
\newcommand{\ccpip}{CC1$\pi^+$\xspace}
\newcommand{\ncpi}{NC1$\pi^0$\xspace}
\newcommand{\ccqe}{CCQE\xspace}
\newcommand{\mares}{\ensuremath{M_A^\mathrm{res}}\xspace}
\newcommand{\ppi}{\ensuremath{|\mathbf{p}_{\pi^0}|}\xspace}
\newcommand{\mb}{MiniBooNE\xspace}
\newcommand{\minerva}{MINER\ensuremath{\nu}A\xspace}
\newcommand{\neut}{\textsc{neut}\xspace}
\newcommand{\nuance}{\textsc{nuance}\xspace}
\newcommand{\tmu}{\ensuremath{T_{\mu}}\xspace}
\newcommand{\pmu}{\ensuremath{|\textbf{p}_{\mu}|}\xspace}
\newcommand{\cost}{\ensuremath{\cos{\theta_{\mu}}}\xspace}
\newcommand{\enu}{\ensuremath{E_{\nu}}\xspace}
\newcommand{\qq}{\ensuremath{Q^{2}}\xspace}
\newcommand{\qqqe}{\ensuremath{Q^{2}_{\textrm{QE}}}\xspace}
\newcommand{\pf}{\ensuremath{p_{F}}\xspace}
\newcommand{\eb}{\ensuremath{E_{b}}\xspace}
\newcommand{\carb}{C\ensuremath{^{12}}\xspace}
\newcommand{\oxy}{O\ensuremath{^{16}}\xspace}
\newcommand{\ie}{i.e.\xspace}
\newcommand{\eg}{e.g.\xspace}
\newcommand{\ma}{\ensuremath{M_{\textrm{A}}}\xspace}
\newcommand{\maqe}{\ensuremath{M_{\textrm{A}}^{\textrm{QE}}}\xspace}
\newcommand{\numu}{\Pnum}
\newcommand{\nue}{\Pnue}
\newcommand{\numubar}{\APnum}
\newcommand{\nuebar}{\APnue}
\newcommand{\enuqerfg}{\ensuremath{E^{\nu}_{\textrm{QE,RFG}}}\xspace}
\newcommand{\enuqe}{\ensuremath{E^{\nu}_{\textrm{QE}}}\xspace}
\newcommand{\chisq}{\ensuremath{\chi^{2}}\xspace}
\newcommand{\chisqmin}{\ensuremath{\chi^{2}_{\textrm{min}}}\xspace}
\newcommand{\chtwo}{CH\ensuremath{_{2}}\xspace}
\newcommand{\wroclaw}{Wroc{\l}aw\xspace}
\newcommand{\km}{\kilo\meter\xspace}
\newcommand{\m}{\meter\xspace}
\newcommand{\evsq}{\eV\ensuremath{^{2}}\xspace}
\newcommand{\POD}{P{\O}D\xspace}
\newcommand{\ecal}{ECal\xspace}
\newcommand{\ecals}{ECals\xspace}
\newcommand{\dsecal}{Ds-ECal\xspace}
\newcommand{\vol}[4]{\ensuremath{#1\times#2\times\unit{#3}{#4}}\xspace}
\newcommand{\area}[3]{\ensuremath{#1\times\unit{#2}{#3}}\xspace}
\newcommand{\pizero}{\pi^{0}\xspace}
\newcommand{\kg}{\kilo\gram\xspace}
\newcommand{\lep}{\ell}
\newcommand{\mnn}{multi-nucleon--neutrino\xspace}
\newcommand{\elt}{\ensuremath{E_{<}}\xspace}
\newcommand{\egt}{\ensuremath{E_{>}}\xspace}


\renewcommand\Im{\operatorname{Im}}

\graphicspath{{figures/}}

\newif\ifpdf
\ifx\pdfoutput\undefined
   \pdffalse
\else
   \pdfoutput=1
   \pdftrue
\fi
\ifpdf
   \usepackage{graphicx}
   \usepackage{epstopdf}
   %\DeclareGraphicsRule{.eps}{pdf}{.pdf}{`epstopdf #1}
   \pdfcompresslevel=9
\else
   \usepackage{graphicx}
\fi

\graphicspath{{figs/}}

\title{GeantReweight \\ 
   \large A Framework for Pion Scattering Reweighting}

\date{}
\begin{document}


\author[1]{Jake Calcutt}
%\author[1]{Kendall Mahn}
\affil[1]{Michigan State University}

\maketitle
\thispagestyle{fancy}
%\linenumbers
%\begin{abstract}

\section{Geant4 Simulation Technique}
\subsection{Tracking and Stepping}
The main philosophy of Geant4's simulation technique is that of 'tracking'. In this, particles are transported through media in a series of steps, possibly interacting along the way. Each G4Event starts with a set of G4Tracks created from a configurable source. These G4Tracks are dynamic objects whose properties are modified while a manager handles the stepping through a medium. Additionally, they are not containers of multiple G4Steps. Rather, a given G4Step can be thought of existing temporarily, and it is where the main action of tracking occurs. It is up to the user to handle the step information (by accessing G4StepPoints) before and after the stepping. 

A G4SteppingManager handles the stepping. It stores a set of active processes which are instantiated by the user at runtime(FOOTNOTE: CAN BE TURNED ON/OFF THROUGHOUT RUN), thus allowing users to customize which processes are simulated. These processes are implemented in various layers of abstraction, starting at the base virtual class G4VProcess. The second layer is comprised of 3 main 'types' of processes - Rest, Continuous, and Discrete - and combinations of these (i.e. ContinuousDiscrete). The processes continue in varying amounts of abstraction until they are fully defined. The main distinction of the Rest, Continuous, and Discrete processes are whether they have the "DoIt" and "GetPhysicalInteractionLength" defined to occur AtRest, Along-step, or Post-step. Rest processes only occur while the particle is at rest and will largely be ignored for the rest of the discussion. The active Continuous and Discrete processes are handled with the following algorithm:
\begin{enumerate}
	\item Each Continuous and Discrete process propose an interaction length.The smallest interaction length ($L_{min}$) is chosen(FOOTNOTE: THIS LEADS TO THE MOST CHALLENGE IN CONCEPTUALIZING REWEIGHTING FOR GEANT AND WILL BE ADDRESSED IN SECTION NUMBER)
	\item "Safety" ($S$) - the distance to the next boundary - is calculated. For charged particles without a field or neutral particles, this is a straight line. Charged particles in fields get a $S$ equal to the arc-length of their path as affected by the field accordingly.
	\item If $L_{min} > S$, the distance to the next boundary is re-calculated. 
	\item The smaller of $L_{min}$ and $S$ is taken as the step length. 
	\item All active continuous processes are invoked along the step. The particle's Kinetic Energy will be updated after all invoked processes are completed. The change in Kinetic Eneregy will be the sum of all the contributions.
	\item The G4Track's properties(Kinetic Energy and 4-position) are updated before discrete processes are invoked. At this point, secondary particles created by continuous processes (i.e. ionization electrons) are stored to be given to the G4TrackingManager. 
	\item Check if it is now at rest, the AtRest procs will be sampled next step. (CHECK THIS. Look in fStep->UpdateTrack?)
	\item If $L_{min}$ was given by a discrete process, it is now invoked.
	\item Track properties are updated. Secondaries are stored.
	\item Check if the discrete process terminated the track.	
	\item The step is finished.
\end{enumerate}

MAYBE SOME MORE STUFF HERE ABOUT TRACKING

\subsection{Choosing a Process}
As stated in the previous section, each active Continuous and Discrete process propose an interaction length before their invocation. The smallest of these interaction lengths is used as the step length. Additionally, if a Discrete process proposed it, that process is invoked after all of the Continuous processes. 

The Discrete processes do not throw a random number at each step as one would usually expect. Rather, each Discrete process samples the distribution 
\begin{equation}\label{eq:1}
P(N) = e^{-N}
\end{equation}
at the beginning of a single track and after each time that process is invoked\footnote{Discrete processes can be furthered categorized depending on if they kill a track (such as pion-inelastic scattering) or the track is still alive after the process is invoked (hadron-elastic scattering). This will affect the treatment of weights slightly, and will be discussed further.\label{fn_Discrete}}. Within one step, the interaction lengths are given by 
\begin{equation}\label{eq:2}
L_p = N_p * \lambda_p
\end{equation}
where $\lambda_p$ is the Mean Free Path of process $p$ (according to the particle's kinematics and, if applicable, its target). These interaction lengths are compared to each other, the interaction lengths of the Continuous processes, and the distance to the next boundary, and the smallest is chosen. Then, each Discrete process that was not invoked has its $N_p$ reduced by $\frac{L_s}{\lambda_p}$, where $L_s$ is the chosen step length. If any Discrete process was chosen - and if that process does not kill the track - then it is not reduced, and is instead given a newly sampled $N_p$ during the next step. In practice, this results in a static \textit{number of interaction lengths} for a given process, where those interaction lengths are constantly updated (changes of kinematics or target).

It is important to realize the interplay between step sizes and Mean Free Paths. Each process has the same chance of throwing a certain number according to \eqref{eq:1}. However, two processes might throw the same number (unlikely, though instructive), but the deciding factor in a process occuring is based on its Mean Free Paths and the step sizes over the particle's trajectory. This is easily realized when considering $N$ as such:
\begin{equation}\label{eq:3}
N = \sum\limits_{s={steps}} \frac{L_{s}}{\lambda_{s}}
\end{equation}
or equivalently,
\begin{equation}\label{eq:4}
N = \sum\limits_{s={steps}} L_{s} * \sigma_{s}
\end{equation}
From this, we can modify \ref{eq:1} as such:
\begin{equation}\label{eq:5}
P(\set{L_s} | \set{\sigma_s}) = e^{-\sum L_s * \sigma_s}
\end{equation}




\section{Weighting Scheme}
From here, we can start moving toward a reweighting framework for pion interactions. We should keep in mind that reweighting is essentially \textit{determining the probability for the same thing to happen under a varied model}. Weights will be assigned when Discrete processes occur, and will change based on the context of what happens. Additionally, as noted in footnote \ref{fn_Discrete}, there will be some divergence in the treatment of Discrete processes that preserve or kill the track. For now, we only consider the processes that kill the track. A singular weight   will be calculated and applied to the track based on the steps it took and the particle's fate when the track is killed.

\subsection{Particle Fates}

We can think of the simulated particle as having two possible fates: it either undergoes an interaction or "survives" after traveling some distance. However, the definition of "survival" needs clarification. In Geant4, the particle will always do \textit{something}, so we should distinguish the fate based on the process that is invoked. Survival processes should then include the pion exiting the world volume (as it is no longer simulated), as well as decaying at rest or being captured at rest. We count the last two as survival processes because they are previously-described AtRest processes which take precendence only once the particle is stopped. Essentially, they require the pion to move a certain distance without interacting, and should be counted as survival processes. 

All other processes are considered interacting processes (FIX THIS WORDING). This includes decaying in flight, as it is treated similar to the rest of the discrete processes in the sampling scheme described above. 

\subsection{Assigning Weights}

Now that we have distinguished the interacting and survival processes, we can move on to calculating the weights that should be applied to the tracks. Whether the tracks survive or interact, they receive a weight that effectively changes the probability to have traveled a specific distance - the track's length - without interacting. Classically, this is the same distribution as in Equation \ref{eq:5}, and should be used in this context. A surviving track will then receive the following weight: 

\begin{equation}\label{eq:surv_weight}
  W_{surv} = \frac{e^{-\sum L_s  \sigma_{tot,s}'}}{e^{-\sum L_s  \sigma_{tot,s}}}
\end{equation}

This will change the distribution of track lengths according to the change in the total cross section\footnote{$\sigma_{tot,s}'$ represents the variation of the total cross section, where any or all of the individual cross sections have been varied.}.

Interacting tracks, however, need to be distinguished according to which type of interaction occured (i.e. inelastic scatter vs. decay-in-flight). Normally, this is done using the individual cross sections. The probability for a specific interaction ($a$) to occur given any interaction happened is $\frac{\sigma_a}{\sigma_{tot}}$. However, to account for the cross section changing throughout the particle's travel, we use the step-length-averaged cross sections. Thus, we give this additional factor\footnote{The $'$ in the numerator of the first term is there to account for the case that the interaction that occured is weighted. If that interaction was not weighted, only the total cross section would be varied.} to tracks that underwent interaction $a$:

\begin{equation}
\frac{R'}{R} = \big(\frac{\sigma'_a}{\sigma'_{tot}}\big)/\big(\frac{\sigma_a}{\sigma_{tot}}\big)
\end{equation}

Making the total weight applied to interacting tracks:

\begin{equation}\label{eq:int_weight_pop}
  W_{int} = \big(\frac{\sigma'_a}{\sigma'_{tot}}\big)/\big(\frac{\sigma_a}{\sigma_{tot}}\big) * \frac{e^{-\sum L_s  \sigma_{tot,s}'}}{e^{-\sum L_s  \sigma_{tot,s}}}
\end{equation}

\subsection{Track-Preserving Processes - Elastic Scattering}
Certain discrete processes, such as elastic scattering, preserve the track within the simulation run. This means multiple elastic scatters could occur before the track is killed by a different discrete process or if it leaves the tracking volume. To account for this, we keep the calculation of the elastic weight separate from the track-killing processes above. We leave out the elastic cross sections from the calculation of the total cross sections used in the previous section, and we assign multiple elastic weights to the track. In essence, we have to assign a weight whenever another number is resampled after the elastic process occurs. 

The total elastic weight will always contain a term in the form of equation \ref{eq:9}. If no elastic scatter occurs, the steps included in the sum span the entire track. For any other number of elastic scatters, those steps span to the end of the track from the step immediately after the last elastic scatter. This accounts for the fact that an elastic scatter did not occur in that range. 

In addition to that 'elastic-survival' factor, multiple weights are assigned from the start of the track to the first elastic scatter and then between any subsequent elastic scatters. These take a similar form to equation \ref{eq:int_weight_pop}, but with only a ratio of the varied and nominal individual cross sections.

The resulting weight then takes the form of:
\begin{equation}\label{ref:elastFull}
W_{elast} = (\prod \limits_{elast} \frac{\sum L_s \sigma_s'}{\sum L_s \sigma_s} \frac{e^{-\sum L_s  \sigma_{s}'}}{e^{-\sum L_s  \sigma_{s}}})(\frac{e^{-\sum L_s \sigma_{s}'}}{e^{-\sum L_s \sigma_{s}}})
\end{equation}
Where the $\sigma_s$ corresponds to the cross section for a given elastic-type process at a given step in the track. Note that every such process that is varied would receive this weight.

\section{Testing and Validation}

\subsection{Thin Target Scattering}
A simulation of pions scattering off a thin target of Liquid Argon was used to extract the pion-Argon scattering cross section. This was motivated by a master's thesis written by a student on the LArIAT experiment\footnote{Irene Nutini - Study of charged particles interaction
processes on Ar in the 0.2 - 2.0 GeV
energy range through combined
information from ionization free charge
and scintillation light}. The thin target was a disk of .5cm in thickness and 1.5m in radius. Monoenergetic beams\footnote{{50, 100, 150, 200, 250, 300, 400, 500, 600, 700, and 800} MeV } of 1E6 pions were sent toward the target and tracked through the volume. 

The number of elastic scatters and the final fate (i.e. Inelastic Scatter, Decay, Transportation\footnote{Leaving the tracking volume}) of each pion was recorded to determine if an interaction occured. The rate of inelastic scatters ($N_{inel}$)and the number of incident pions ($N_{inc}$) were used to determine the reactive cross section at each energy: 

\begin{equation}\label{ref:reactive_xsec}
\sigma_{reac} = \frac{N_{inel}}{N_{inc}}\frac{1}{Nx}
\end{equation}
Additionally, the total cross section was determined by including the rate of elastic scattering:
\begin{equation}\label{ref:total_xsec}
\sigma_{total} = \frac{N_{scat}}{N_{inc}}\frac{1}{Nx}
\end{equation}
Here, $N_{scat}$ is defined as the number of pions that had any amount of elastic scatters within the volume or whose final fate was an inelastic scatter. 

The simulation was ran with both the nominal inelastic and elastic scattering cross sections, as well as with 3 sets of variations consisting of scaling the inelastic and elastic cross sections separately. These sets are detailed in Table \ref{ref:variations}. 

\begin{center}\label{ref:variations}
  \begin{tabular}{| c | c  c |}
  \hline
  Set & Inelastic Scale & Elastic Scale  \\
  \hline
  1 & 1.5 & 1. \\ 
  \hline
  2 & 1.  & 1.5 \\
  \hline	
  3 & 1.5 & 1.5 \\
  \hline  
  \end{tabular}
\end{center}

Figures ... show the reactive cross sections produced by the nominal and varied simulations, as well as with weighting applied to the nominal simulation. They also compares these to the reactive cross section as extracted by Nutini.

\begin{center}
\includegraphics[width=.75\textwidth]{{/home/jake/comp_pdfs/reactive_thin_xsec_inel1.5_elast1}.pdf}
\includegraphics[width=.75\textwidth]{{/home/jake/comp_pdfs/reactive_thin_xsec_inel1_elast1.5}.pdf}
\includegraphics[width=.75\textwidth]{{/home/jake/comp_pdfs/reactive_thin_xsec_inel1.5_elast1.5}.pdf}
\end{center}

Figures ... show the total cross sections produced by the nominal and varied simulations, as well as with weighting applied to the nominal simulation. They also compares these to the total cross section as extracted by Nutini.

Figures ... show the ratio between the weighted and varied samples for both the reactive and total cross sections


\begin{center}
\includegraphics[width=.75\textwidth]{{/home/jake/comp_pdfs/total_thin_xsec_inel1.5_elast1}.pdf}
\includegraphics[width=.75\textwidth]{{/home/jake/comp_pdfs/total_thin_xsec_inel1_elast1.5}.pdf}
\includegraphics[width=.75\textwidth]{{/home/jake/comp_pdfs/total_thin_xsec_inel1.5_elast1.5}.pdf}
\end{center}


\begin{center}
\includegraphics[width=.75\textwidth]{{/home/jake/comp_pdfs/thin_ratio_inel1.5_elast1}.pdf}
\includegraphics[width=.75\textwidth]{{/home/jake/comp_pdfs/thin_ratio_inel1_elast1.5}.pdf}
\includegraphics[width=.75\textwidth]{{/home/jake/comp_pdfs/thin_ratio_inel1.5_elast1.5}.pdf}
\end{center}
\section{Edits to Geant4.10.03.p03}

As well as creating the simulation and reweighting packge, I also made edits to Geant4's base code. The edits are summarized here:
\begin{enumerate}
\item Added two extra physics list based off of the G4HadronPhysicsFTFP\_BERT and G4HadronElasticPhysics lists. 
	\begin{enumerate}
	\item These allow the user to specify - at run-time - a factor that scales the inelastic and elastic pion cross sections, respectively. 
	\item Additionally, had to add versions of respective 'particle builder' and 'list constructor' classes. 
	\end{enumerate}
\item Implemented methods in various physics process classes that extract the mean free path of the particle in its current state.
	\begin{enumerate}
	\item The original methods that do this were private within their respective classes, and so could not be called by higher levels of Geant.

	\end{enumerate}
\item Edited the G4SteppingManager to handle extracting information about active processes (name and Mean Free Path) at each step.
\end{enumerate}

%\appendix
%\section{Interaction Probability Derivation}
%the derivation for the probability of interaction given a cross section
%
%This is essentially the probability for a particle to travel distance L before interacting
%$
%dN = -N \sigma dL \\ 
%\int\frac{dN}{N} = -\int \sigma dL \\
%log(\frac{N}{N_0}) = - \sigma L \\
%\frac{N}{N_0}dL = e^{-\sigma L}dL \\
%$

%\subfile{DUNE_appendix_eff_plots.tex}

%\begin{thebibliography}{7}

%\bibitem{DUNE_CDR1}
%The DUNE Collaboration
%\textit{Long-Baseline Neutrino Facility (LBNF) and Deep Underground Neutrino Experiment (DUNE) Conceptual Design Report Volume 1: The LBNF and DUNE Projects}
%arXiv:1601.05471v1
%
%
%\end{thebibliography}

\end{document}




